\documentclass[a4paper,12pt]{exam}

\usepackage[utf8]{inputenc}
\usepackage[ngerman]{babel}
\usepackage[T1]{fontenc}
\usepackage{graphicx}
\usepackage[cache=false]{minted}

\begin{document}
\begin{titlepage}
    \begin{center}
        \vspace*{1cm}
        \includegraphics[width=12cm]{Logo.png}
        
        \textbf{\huge Santa Claus Problem}
        
        \vspace{0.5cm}
        NVS Projekt 1
                 
        \vspace{1.0cm}
    
        \textbf{Alexander Grill 5CHIF}
        
        \today
        
        \vfill
                 
                 
        \vspace{0.5cm}
                 
        Informatik\\
        HTBLUvA Wr.Neustadt\\
        Österreich\\

                 
    \end{center}
\end{titlepage}  


\newpage
\tableofcontents
\newpage
\section{Einführung}
\subsection{Vorwort}
Auf Grund der akuellen Lage(COVID-19) in Österreich, wurde der Unterricht an den Schulen
in Form von Distance-Learning abgehalten. Dehsalb war die Durchführung der Praktischen Arbeit in
Netzwerktechnick nicht möglich. Herr Professor Kolouseck gab uns daraufhin eine Projektarbeit die bis zum 
20.01.2021 zu erledigen ist. Die Gesamtnote des Projekt ersetzt ausschließliche die Note,
die man bei der Praktischen Arbeit erworben hätte In diesem Projekt geht es darum mit Prozessen, Thread und Synchronisation
für die jeweiligen Anwendungszenarien entwickelt werden, um damit zu zeigne, dass die Praktischen
Fähigkeiten zum Implementieren verteiler Systeme erworben wurden.

\subsection{Erklärung der Grundproblematik}


\begin{minted}{cpp}
\end{minted}




\end{document}