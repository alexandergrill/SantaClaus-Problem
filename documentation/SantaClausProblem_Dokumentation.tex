\documentclass[a4paper,12pt]{exam}

\usepackage[utf8]{inputenc}
\usepackage[ngerman]{babel}
\usepackage[T1]{fontenc}
\usepackage{graphicx}
\usepackage[cache=false]{minted}
\usepackage{dirtree}

\begin{document}
\begin{titlepage}
    \begin{center}
        \vspace*{1cm}
        \includegraphics[width=12cm]{Logo.png}
        
        \textbf{\huge Santa Claus Problem}
        
        \vspace{0.5cm}
        NVS Projekt 1
                 
        \vspace{1.0cm}
    
        \textbf{Alexander Grill 5CHIF}
        
        \today
        
        \vfill
                 
                 
        \vspace{0.5cm}
                 
        Informatik\\
        HTBLUvA Wr.Neustadt\\
        Österreich\\

                 
    \end{center}
\end{titlepage}  


\newpage
\tableofcontents
\newpage

\section{Einführung}
 In diesem Kaptiel wird erklärt was der Grund für die Umsetzung war, welche Punkte in die Benotung miteinfließen und um welche Probelmstellung es sich handelt. Die genauere Erläuterung
 der Grundproblematik wird aber im 2 Kaptiel beschrieben.
\subsection{Vorwort}

Auf Grund der akuellen Lage(COVID-19) in Österreich, wurde der Unterricht an den Schulen
in Form von Distance-Learning abgehalten. Dehsalb war die Durchführung der Praktischen Arbeit in
Netzwerktechnick nicht möglich. Herr Professor Kolouseck gab uns daraufhin eine Projektarbeit die bis zum 
20.01.2021 zu erledigen ist. Die Gesamtnote des Projekt ersetzt ausschließliche die Note,
die man bei der Praktischen Arbeit erworben hätte. In diesem Projekt geht es darum mit Prozessen, Thread und Synchronisation
für die jeweiligen Anwendungszenarien entwickelt werden, um damit zu zeigne, dass die Praktischen
Fähigkeiten zum Implementieren verteiler Systeme erworben wurden. Die Programmiersprache die für
die Implementierung verwendet wird, ist C++.

\begin{description}
    \item[Das Projekt besteht aus 3 wichtigen Punkten:] ~\par
    \begin{itemize}
        \item praktischen Ausarbeitung
        \begin{itemize}
            \item  dabei wird die Grundproblematik simuliert, klargestellt und verschiedenste Szenarios angewendet 
        \end{itemize}
        \item theoretisch Ausarbeitung
        \begin{itemize}
            \item  in diesem Teil wird die Aufgabenstellung, Probleme, Dokumentation der Source Codes usw. festgehalten 
        \end{itemize}
        \item Git Hub Repository
        \begin{itemize}
            \item das Projekt muss auf GitHub gehostet werden, um die Verwaltung der Projekts zu erlechtern und, um den Workflow zu dokumentiern 
            \item Commites müssen gemacht werden, das man etwas als Patch verwenden kann (Fehler, Probleme können zurückgenommenr werden)
            \item Commit-Meldungen sollen kurz und prägnant sein und sollen ausdrücken für was dieser Commit steht.
        \end{itemize}
    \end{itemize} 
\end{description}

\subsection{Motivation}
In diesem Projekt wird die Santa Claus Problematik erläutert und mit Hilfe eines C++ Projekt simuliert. Santa Claus wird wegen 
zwei Faktoren geweckt, sonst benötigt er umbedingt seinen Schlaf. Er wird geweckt, wenn Elfen ihm brauchen, weil sie mit der Arbeit
nicht weiter kommen, und dadurch die Produktion der Geschenke für die Kinder verlangsamt oder gar gestoppt wird. Deshalb ist es wichtig, dass 
in diesem Moment Santa geweckt wird, um Santa den Elfen helfen zu können. Zugleich kommen auch die Rentiere in unterschiedlichen Zeitpunkten zurück von ihrere Reise und sammeln sich im Stall.
Dort warten sie gemeinsam bis sie vollzählig sind und Santa mit ihnen, folgedessen die Geschenke zu den Kindern liefern. Auch in diesem Moment ist es
äußerst wichtig, dass Santa geweckt wird, weil er schon einen Teil der Geschenke mit den Rentieren ausliefern kann. Zu beachten ist jedoch, dass es zu keinem Zusammenstoß zwischen den Gruppen kommmt
und Santa Claus nicht weiß, welche Tätigkeiten zuerst vollendet werden sollen bzw. welche eine höhere Priorität haben.
\newpage

\section{Aufgabenstelllung}
Dieses Kapitel umfasst die genaue Beschreibung der Grundproblematik und deren Aufgabenstelllung. Weiters wird auch über die Idee der Umsetzung geschrieben. Zusätzliche Erweiterungen wie
Kommandozeilenparameter, Konsolenausgabe etc.
\subsection{Erleuterung der Grundproblematik}
Dieses Problem stammt aus William Stalling Operating Systemts. Dabei wird folgende Problemstellung beschreiben:\\ 
\\Santa Claus sitzt in seinem Spielwarenladen am Nordpol und schläft, während seine zurückgekommene Rentiere im Stall fressen
um Kräfte für die jährliche Auslieferung der Geschenk an die Kinder zu Weihnachten zu sammeln. Seine fleißigen Elfen arbeiten sorgfälltig an den Geschenken der Kinder
in der Spielzeugfabrik. Hin und wieder kann es vorkommen, dass die Elven beim Basteln vor einem Problem stehen und ohne Hilfe vom Santa nicht mehr weiter machen können.
Es wäre eine Katastrophe, wenn die Geschenke nicht rechtzeitig am Heiligen Abend an die Kinder ausgeliefert werden können, weil einige Elfen die Produktion der Geschenke blockiert haben. \\\\
Aus diesem Grund muss Santa Claus umbedingt geweckt werden, obwohl er ein gewisses Maß an dringendem Schlaf benötigt. Erst wenn 3 oder mehr Elfen ohne Hilfe nicht mehr weiter arbeiten können wecken sie ihn auf.
Wenn drei Elfen ihr Problem gelöst haben, müssen alle anderen Elfen, die den Weihnachtsmann besuchen wollen auf die Rückkehr dieser Elfen warten.
Jedoch ist zu beachten, dass wenn Elfen vor der Tür seines Ladens warten, während das letzte Rentier für die Auslieferung aus dem Tropen zurückkehrt, beschließt der Weihnachtsmann, die Elfen bis nach Weihnachten warten zu lassen. Denn die es ist wichtiger den Schlitten
fertigzustellen und die Packete auszuliefern.
Nachdem Santa Claus ihnen beim Problem geholfen hat, können die Elfen erleichternd weiterbasteln und Santa Claus seinen Schlaf forsetzen.\\
\\Allerdings möchte Santa Claus die Kinder nicht zu lange auf ihre Geschenke warte lassen, und deshalb soll er auch aufgeweckt werden, wenn genuge Rentiere bereit sind bzw. zurück aus ihrem Urlaub im Südpazifik gekommen sind, um den 
Schlitten mit den Geschenken zu ziehen und eine Ladung Geschenke zu verteilen. Das letzte Rentier, das ankommt, muss den Weihnachtsmann holen, währen die anderen Rentiere gemütlich in der warmen Hütte warten, bevor sie vor dem Schlitten gespannt werden.\\
\\Die restliche Zeit kann Santa schlafen, um für die nächste anstregenden Tätigkeiten Kräfte zu schöpfen. Mann kann davon ausgehen, dass stets genug Arbeit für die Elfen und genug Geschenke für eine Ladung Geschenke vorhanden sind.
Das heißt Elfen und Rentiere sind weitgehend unabhängig voneinander. Santa Claus, seine Rentiere und die Elfen müssen jeweils durch einen eigene Thread umgesetzt werden, sodass das unterschedlichen Eintrehten von Szenarien festgehalten werden kann.
\newpage

\subsection{Idee}
Das Programm besteht aus drei Threads nämlich: SantaClaus, Rentiere, Elven. Der Thread SantaClaus schläft so lange bis, entweder alle seine benötigten Rentier zurückgekommen sind, oder 3 oder mehrere Elven ihm drigend brauchen.
Im Thread Renntier wird modiliert, dass noch einer zufälligen Dauer, Rentiere zurückkommen und Santa danach aufwecken, sodass sie schon einen Teil der Geschenke an die Kinder liefern können. Im Thread Elven wird ebenso modiliert, dass
nach einer zufälligen Zeit Elven Santa um seine Hilfe bitten. Nach einer bestimmten Anzahl von Elfen wird auch hier Santa geweckt, sodass sie weiter arbeiten können. Das wichtige ist vorallem, dass die Geschenke bis zum Heiligen Abend
ausgeliefert werden können und, dass wenn alle Rentier da sind, Santa auf die Hilferufe der Elfen verzichtet und die Geschenke ausliefert.

\subsection{Kommandozeilenparameter}
Mit Hilfe von Kommandozeilenparameter soll dem Benutzer ermöglicht, werden die Anzahl der Renntier und der Elfen zu definieren. Diese zwei Zahlen legen die maximal benötigte Anzahl fest, um den Thread Santa Claus aus seinem Schlaf zu holen.
Ebenso kann der Benutzer auch die Zeit in Stunden bis zu Weihnachten angeben, dadurch kann geprüft werden ob es sich zu geringer Zeit ausgeht die Packete bis zu Weihnachten auszuliefern.
Es werden keine negativen Zahlen, Buchstaben, Sonderzeichen akzeptiert und wenn der Benutzer sonstige Hilfe braucht kann er sich die Informatione mit dem Kommandozeilenparameter holen.
\begin{minted}{c}
Santa Claus Problem
Usage: ./santa_claus_problem [OPTIONS]

Options:
  -h,--help             Print this help message and exit
  -r,--r INT            number of reindeer, which will be needed to fly
  -e,--e INT            number of elves, that work in the factory
  -t,--t INT            number of hours until christmas
  -j,--j TEXT:FILE      write santa, reindeer, elves details in json File
  -d,--d                show you a table about the Objects Santa, Elves, Reindeer
\end{minted}

\subsection{Konsolenausgabe}
Die Abarbeitung bzw. Resultate der Threads SantaClaus, Elven und Rentiere werden in der Kommandozeile ausgegeben. Somit kann der Benutzer gut nachfollziehen, welche Tätigkeiten das Programm abgewickelt hat, welche noch bevor stehen und
ob es zu Fehlersituation oder Ausnahmen gekommen ist. Angenommen es kommen Rentier zurück oder Elven benötigen Santa´s Hilfe so wird auch in der Kommandozeile dies wird mit spdlog festgehalten. Um am Ende einen genauen und übersichtlichen 
Überblick den Benutzer zu verschaffen, wird mit dem Kommandozeilenparameter -tab eine Tabelle ausgegeben die Zeigt, wie viele Rentiere im Stall sind, wie viele Elfen Hilfe brauchten und wie viele Stunden Santa Claus geschlafen hat.
\newpage

\section{Themenbereiche}
Das Projekt umfasst folgende Themengebiete, die unter anderem in folgenden Folien gut dokumentiert sind und von denen ich mir einige Tipps geholt habe:
    \begin{itemize}
        \item  10\_processes
        \item  11\_threads
        \item  12\_threads2 
        \item  13\_synchronization 
        \item  14\_condition\_synchronization 
        \item  15\_sync\_mechanisms 
        \item  16\_threadsafe\_interfaces 
        \item  17\_dist\_sync 
        \item  18\_task\_based\_programming 
        \item  19\_parallel\_programming 
        \item  20\_threads\_perfmem
        \item  21\_encoding
        \item  22\_data\_interoperability
        \item  23\_character\_encoding
    \end{itemize} 
\newpage
\section{Implementierung}
In diesem Kapitel geht es grundsätzlich um die technische Realisierung der Aufgabenstellung. Im folgendem Abschnitt wird auch
der Aufbau des Projekts beschrieben. Außerdem enthält dieses Kapitel auch die Dokumentation des Source Codes und wichtige Informationen bezgüglich Bibliotheken, 
die im Projekt verwendet wurde. 

\subsection{Aufbau}
Das Projekt wurde in 2 wesentlich Verzeichnisse eingeteilt, nämlich in source und documentation. In dem Verzeichniss documentation sind alle Unterlagen der theoretisch Arbeit 
abgelegt worden. Im Verzeichniss source befinden sich alle relevanten Datein der technischen Umsetzung der Aufgabenstellung.
Die drei erwähnten Objekte wurden zu je 3 Module aufgeteilt.
Das heißt die Klassendefinition, Konstruktoren und Methodenprototypen von Santa Claus, Elven und Renntiere stehen in der jeweiligen h-Datei. Die Methodendefinitionen, der dazugehörigen Klasse,
wurden in der dazugehörigen cpp-Datei kodiert. Zusätzlichen Funktionen die für die Umsetzung des Projekts dringed notwendig waren, wurden in dem Modul utils declariert und definiert.
Da einige Source Code Abschnitte nicht verständlich und nachfollziebar sind, wurden alle Variablen, Funktionen, Klassen, Methoden und sonstiges gut dokumentiert.\\
\dirtree{%
.1 documentation\DTcomment{enthält alle Datein der theoretisch Ausarbeitung}.
.2 {Logo.png}.
.2 {SantaClausProblem\_Dokumentation.pdf}.
.2 {SantaClausProblem\_Dokumentation.tex}.
.1 source\DTcomment{enthält alle Datein der praktischen Ausarbeitung}.
.2 build\DTcomment{darin befinden sich alle automatisch generiert werden Datein}.
.3 {build.ninja}.
.3 {compile\_commands.json}.
.3 {meson\-info}.
.3 {meson\-logs}.
.3 {meson-private}.
.3 {santa\_claus\_problem}.
.3 {santa\_claus\_problem@exe}.
.3 {santa\_problem.json}.
.2 include\DTcomment{darin befinden sich alle h-Dateien}.
.3 {Elves.h}.
.3 {Reindeer.h}.
.3 {SantaClaus.h}.
.3 {utils.h}.
.2 src\DTcomment{darin befinden sich alle cpp-Dateien}.
.3 {Elves.cpp}.
.3 {Reindeer.cpp}.
.3 {SantaClaus.cpp}.
.3 {utils.cpp}.
.3 {main.cpp}.
.2 {meson.build}.
.2 {meson\_options.txt}.
}
\subsection{Source Codes Dokumentation}
In diesem Abschnitt werden einzelne Programmteile beschreiben und genau erläutert, welche Aufgabe sie haben und welche Resultate davon entzogen werden.
Es wird auch auf einzelne Grundgedanken, Umsetzungen und Lösungsvarianten daraufeingegangen, um 
\subsubsection{Elves [Klasse]}
\begin{minted}{cpp}
//Klasse Elven
class Elves{
private:
//Variablen
    SantaClaus *sc;     //Verweis auf das dazugehörige SantaClaus Objekt
    std::mutex &mxe;    //Mutex Obekt
    int elves{0};       //Elfen
    int maxelves{0};    //Elven die benötigt werden um Santa zu wecken
    int elvessum{0};    //alle Elfen die Hilfe benötigten 
public:
//Condition Variable
    std::condition_variable elfTex;
//Konstruktor
    Elves(int me, std::mutex& xe):mxe{xe}{
        maxelves = me;
    }
//Methoden
    //Arbeitsablauf der Elfen, Santa wird geweckt wenn Elfen in benötigen
    void tinker();
    //Santa Claus hilft jeden einzelnen Elfen
    void get_Help();
    //gibt die Anzahl der Elfen zurück, die Hilfe brauchen
    int get_Elves();
    /*gibt die Maximale Anzahl der Elfen zurück,
     die benötigt werden um Santa zu wecken*/
    int get_MaxElves();
    int get_SumElves();
    void set_Santa(SantaClaus *s);
};
\end{minted}
\newpage
\subsubsection{Reindeer [Klasse]}
\begin{minted}{cpp}
//Klassen Rentiere
class Reindeer{
private:
//Variablen
    SantaClaus *sc;         //Verweis auf das dazugehörige SantaClaus Objekt
    std::mutex &mxr;        //Mutex Objekt
    int reindeer{0};        //Rentiere
    int maxreindeer;        //Rentiere die benötigt werden um Santa zu wecken
public:
//Condition Variable
    std::condition_variable reindeerSem;
//Konstruktor
    Reindeer(int mr, std::mutex& xr):mxr{xr}{
        maxreindeer = mr;
    }    
//Methoden
    //Rückkunft aller Rentiere aus dem Osten
    void comeback();
    //wenn alle Rentiere da sind, werden sie vom Santa an dem Schlitten angehängt
    void get_Hitched();
    //gibt die Anzahl der Rentier zurück, die zurückgekommen sind
    int get_Reindeer();
    /*gibt die Maximale Anzahl der Rentier zurück, 
    die benötigt werden um Santa zu wecken*/
    int get_MaxReindeer();
    //setzt den Verweis, auf das jeweilige SantaClaus Objekt
    void set_Santa(SantaClaus *s);
    //setzt die Anzahl der Renntier auf 0 ->für Debuggen notwendig gewesen
    void reset_Reindeer();
};
\end{minted}
\newpage
\subsubsection{Santa Claus [Klasse]}
\begin{minted}{cpp}
//Klasse Santa Claus
class SantaClaus{
private:
//Variablen
    Elves &elv;                 //Verweis auf das dazugehörige Elven Objekt
    Reindeer &ren;              //Verweis auf das dazugehörige Renntier Objekt
    std::mutex &mxs;            //Mutex Obekt
    double blithelytime{0};     //gesamte Schlafzeit
    bool doaction{false};       //bool Variable, für santaSem.wait
    bool readytofly{false};     //bool Variable, für ren.reindeerSem.notify_one       
    bool readytohelp{false};    //bool Variable, für elv.elfTex.notify_one
public:
//Condition Variable
    std::condition_variable santaSem;
//Konstruktor
    SantaClaus(Elves& e, Reindeer& r, std::mutex& xs): elv{e}, ren{r}, mxs{xs}{
    }
//Methoden
    /*Santa schläft und wird geweckt, wenn alle Rentiere zurück 
    sind oder Elfen ihm 4brauchen*/
    void sleep();      
    //gibt die Zeit aus, die Santa munter war         
    int get_Blithelytime(); 
    //gibt true, false zurück je nachdem wie viele Rentiere zurück gekommen sind
    bool get_Readytofly(); 
    //gibt true, false zurück je nachdem wie viele Elven Santa benötigen
    bool get_Readytohelp();
    //addiert die Zeit, in der Santa munter war
    void set_Blithelytime(double t);
    //setzt die Variable readytohelp auf false
    void set_Readytohelp();
    //setzt die Variable doaction auf true
    void set_Doaction();
};
\end{minted}
\newpage
\subsubsection{Zusätzliche Funktionen}
\subsubsection{Sonstiges}

\subsection{Verwendete Bibliotheken}
\subsubsection{CLI11}
\subsubsection{spdlog}
\subsubsection{tabulat}
\subsubsection{json}


\section{Anwendungsfälle}
Im folgendem Abschnitt werden alle Andwendungsfälle, die ein Benutzer durchführen kann, verkörpert. Dabei wird besonders auf
die Kommandozeilenparameter eingegangen.
\\
\begin{minted}{cpp}
    ./santa_claus_problem
\end{minted}
Dabei wird das Programm santa\_claus\_problem gestartet. Wobei 9 Rentiere und 3 Elfen
gebraucht werden um Santa Claus aus seinem Schlaf zu wecken. Es werden 24 Stunde bis zur Auslieferung der Geschenke gewährleistet
\\
\begin{minted}{cpp}
    ./santa_claus_problem -h
\end{minted}

\begin{minted}{cpp}
    ./santa_claus_problem -r 3 -e 5 -t 10
\end{minted}

\begin{minted}{cpp}
    ./santa_claus_problem -r 3 -e 5 -t 10 -d
\end{minted}

\begin{minted}{cpp}
    ./santa_claus_problem -r 3 -e 5 -t 10 -d -e santa_problem.json
\end{minted}




\end{document}